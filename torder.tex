\documentclass[11pt]{article}
\author{Nathan C. Sanders \\ ncsander@indiana.edu}
\title{An Efficient Algorithm for Generating T-Orders}
\begin{document}
\maketitle
\begin{enumerate}
  \section{T-Orders}
\item T-Orders were noticed by Anttila (2006) in his work on variation
  in language.

\item
  They can be used to predict variation from some constraint set.
\item The outputs that win the most in the factorial typology are also the ones that
  appear the most in variation.
\item This agrees with the view that language variation arises from permutations
  of a fixed constraint set.
\subsection{T-Order Definition}
\item A T-Order is a directed acyclic graph whose nodes are all possibly winning
  candidates for some input and set of constraints.
\item This `winning set' is an important concept that is useful later.
\item There is an edge from any node whose appearance always implies the
  appearance of another node.
\item Actually, it's the transitive closure of this graph. But who's counting?
\item For example, assume three inputs: /a/, /b/ and /c/.
\item $\forall C , (C(/a/) \mapsto [aa]) \to (C(/b/) \mapsto [bb])$ would
  produce an edge from [aa] to [bb].
\item This notation assumes that $C$ is an ordered composition of
  the constraints in {\sc Con}, using the model of constraints as functions from
  an output set to a smaller output set.
\item One point to notice is that harmonically bounded candidates can never have
  any connecting edges, because an input will never map to them as the winning
  candidate.
\item This is important later, because in some of the methods I'm trying generate
  all edges to and from harmonically bounded candidates and they have to be
  removed in a separate step.
\subsection{T-Order Uses}
\item Uses for t-orders are explained well in Anttila (2006).
\item Essentially, the frequency of items in variation increases down the
  implicational hierarchy formed by the t-order.
\item For example, if [aa] $\to$ [bb] $\to$ [aaa], then [aa] will be
  the least common variant of the three.
\item Somehow this explanation is wrong, because it seems important enough
  to have relations between outputs of disparate inputs.
\item Also I don't think [aa] $\to$ [aaa] is possible since that would mean that
  a dialect with [aa] would also have [aaa]: two outputs for the same input.
  \section{T-Order Factorial Generation}
\subsection{Naive Generation}
\item The naive way to generate a t-order is to evaluate all possible orderings
  of the constraint set and track which outputs win for each input.
\item Anttila and ?'s software works with OTSoft (Hayes 1991 to 2004) to do this.
\item However, this method does not scale past half a dozen constraints or so
  because of the factorial typology that must be calculated.
\item Anything besides $5!$ is {\it terra incognita} for computer
  scientists, so it would be best to avoid factorial completely.
  \subsection{Bounding Sets}
\item To avoid calculating all constraint orders, we need to step back and
  re-arrange the problem.
\item What is really interesting is the possible winners, not the
  factorial typology itself.
\item Winners are defined as those candidates that can win for some arrangement
  of the candidate set. Winners are exactly those candidates that have a node in
  the t-order.
\item Prince and Lodovici provide a fast way to find all possible winners for a
  set of constraints.
\item By splitting the necessary factorial arrangement of constraints into a tree
  structure, less work need be done.
\item I believe the worst case running time is still factorial, but the average case
  is far better. (I need to re-read the section in which running time is analysed.)
\item This is because not all nodes of the tree need to be explored---after a
  certain condition is reached, a node does not have to investigate its children.
\item TODO: Describe the algorithm for constructing the tree here.
\item I have a couple of e-mails that describe it fairly well. (Or from my journal.)
\item Essentially, this arranges the candidates so that each node represents all
  possible winners given only the constraints in play for that node.
\item The additional terminal case of only having one winner means that
  children of that node are uninteresting and all factorial typologies in which the
  remaining constraints are critical will produce the same winner.
\item So no further work needs to be done.
  \section{Extracting T-Orders from Bounding Trees}
\item Since the tree of bounding sets knows for any constraint set which
  candidates are viable, it should be able to produce the same information on
  which candidates win relative to other candidates.
\item One simple way to calculate this is to use all leaf nodes that
  have two winning candidates.
\item There is an edge from any of these winners that bounds any other winner.
\item This method dumps all candidates into the same pool, regardless of
  input form.
\item Unfortunately, this does not work for candidates with identical violation
  profiles.
\item There can still be differences in the factorial typology based on the
  competing candidates.
\item So my most recent attempt generates separate bounding trees for each
  input separately: the candidate set is divided by input and the bounding trees
  are generated in parallel.
\item However, I haven't discovered the correct way to extract edges from this.
\item It is possible to retain the simple test from a grouped bounding tree, but
  the problem is finding edges between candidates with identical violation profiles
  without including spurious edges.
\item Even if the methods are combined then further testing is needed to make
  sure that it has adequate coverage for more examples.
  \section{Conclusion}
\item T-orders are pretty interesting.
\item One thing that makes them interesting is that they involve a factorial process
  of generating a complete typology.
\item A successful implementation must provide a way to handle this in an
  efficient way.
\item Two basic problem-solving methods are useful in this: dividing the work
  up as a tree and using domain knowledge to improve data structure efficiency.
\item Although the current methods don't produce the right answer, they are
  pretty close and pretty fast.
\end{enumerate}
\end{document}
%%% Local Variables: 
%%% mode: latex
%%% TeX-master: t
%%% End: 
