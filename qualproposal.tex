\documentclass[11pt]{article}
\usepackage{robbib}
\author{Nathan Sanders \\ \tt{ncsander@indiana.edu}}
\title{Generation of Factorial Typologies in OT : Four Algorithms}
\begin{document}
\maketitle

In this qualifying paper, I analyze four algorithms for generating
factorial typologies. The operation of each algorithm is explained and
its time and space requirements given. % proven?
A factorial typology is a list of all combinations of outputs that can
win for some ordering of the relevant constraints. Each combination of
output forms for a given set of inputs is called a candidate set.

Factorial typologies are an important part of Optimality Theory
(OT). When constructing a phonological analysis, linguists use a
factorial typology to check the predictions made by the
relevant constraints. Candidate sets that are part of the factorial
typology should be attested in at least one language of the world.
Conversely, candidate sets that are not part of the factorial typology
can never win an optimality theoretic evaluation. Therefore, OT predicts that
these candidate sets should not appear in any
language.

The first algorithm demonstrated is the naive one,
which generates all factorial combinations of constraints and then evaluates
them using OT's {\sc Eval}.  The second is similar, but
takes care to evaluate only once combinations that must produce that same
winners. This algorithm is presented for the first time in this
paper.

The third algorithm is used in Hayes' OT-Soft \cite{hayes03}. Assuming
a finite candidate set, it tests every possible candidate set with
Recursive Constraint Demotion (RCD) \cite{tesar93}. Only candidate
sets for which RCD terminates successfully are part of the factorial
typology. This algorithm avoids the problem of generating all
factorial combinations of constraints. To my knowledge, this is the
first published analysis of this algorithm, although the algorithm is
moderately well-known and is referenced in other work such as
\namecite{pater08}.

The fourth algorithm is based on \quotecite{riggle08} work on
$r$-volumes. Riggle's algorithm to calculate $r$-volumes is similar to
RCD but can terminate faster. This factorial typology algorithm,like
Hayes', evaluates every candidate set, but since its core is the faster
$r$-volume algorithm, it is correspondingly faster. Riggle first
demonstrated the algorithm in a presentation to the Linguistic Society
of America \cite{riggle07}.

\bibliographystyle{robbib}
\bibliography{central}

\end{document}
